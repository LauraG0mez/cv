%%%%%%%%%%%%%%%%%
% This is an sample CV template created using altacv.cls
% (v1.1.5, 1 December 2018) written by LianTze Lim (liantze@gmail.com). Now compiles with pdfLaTeX, XeLaTeX and LuaLaTeX.
%
%% It may be distributed and/or modified under the
%% conditions of the LaTeX Project Public License, either version 1.3
%% of this license or (at your option) any later version.
%% The latest version of this license is in
%%    http://www.latex-project.org/lppl.txt
%% and version 1.3 or later is part of all distributions of LaTeX
%% version 2003/12/01 or later.
%%%%%%%%%%%%%%%%

%% If you need to pass whatever options to xcolor
\PassOptionsToPackage{dvipsnames}{xcolor}

%% If you are using \orcid or academicons
%% icons, make sure you have the academicons
%% option here, and compile with XeLaTeX
%% or LuaLaTeX.
% \documentclass[10pt,a4paper,academicons]{altacv}

%% Use the "normalphoto" option if you want a normal photo instead of cropped to a circle
% \documentclass[10pt,a4paper,normalphoto]{altacv}

% \documentclass[10pt,a4paper,normalphoto,academicons]{altacv}
\documentclass[10pt,a4paper,ragged2e]{altacv}

%% AltaCV uses the fontawesome and academicon fonts
%% and packages.
%% See texdoc.net/pkg/fontawecome and http://texdoc.net/pkg/academicons for full list of symbols. You MUST compile with XeLaTeX or LuaLaTeX if you want to use academicons.

% Change the page layout if you need to
\geometry{%
left=1cm,%
right=8.5cm,%
marginparwidth=7.0cm,%
marginparsep=8mm,%
top=1.25cm,%
bottom=1.25cm}

% Change the font if you want to, depending on whether
% you're using pdflatex or xelatex/lualatex
\ifxetexorluatex
  % If using xelatex or lualatex:
  \setmainfont{Ubuntu}
  % \setmainfont{Lato}
\else
  % If using pdflatex:
  \usepackage[utf8]{inputenc}
  \usepackage[T1]{fontenc}
  \usepackage[default]{lato}
\fi

% Change the colours if you want to
\definecolor{Mulberry}{HTML}{72243D}
\definecolor{SlateGrey}{HTML}{2E2E2E}
\definecolor{LightGrey}{HTML}{666666}
\definecolor{GoodSamaritan}{HTML}{3C6382}
\colorlet{heading}{GoodSamaritan}
% \colorlet{heading}{Sepia}
% \colorlet{accent}{Mulberry}
\colorlet{accent}{GoodSamaritan}
\colorlet{emphasis}{SlateGrey}
\colorlet{body}{LightGrey}

% Change the bullets for itemize and rating marker
% for \cvskill if you want to
\renewcommand{\itemmarker}{{\small\textbullet}}
\renewcommand{\ratingmarker}{\faCircle}

%% sample.bib contains your publications
% \addbibresource{sample.bib}

\begin{document}
\name{Germain Salvato Vallverdu}
\tagline{Researcher in molecular simulations}
\photo{3.4cm}{../IPREM-gvallver_carre_50p.jpg}
% \photo{2.8cm}{../img/gvallver-red.jpg}
\personalinfo{%
  % Not all of these are required!
  % You can add your own with \printinfo{symbol}{detail}
  1983 August 10 -- Married, 2 children\\
  \vspace{1ex}
  \email{germain.vallverdu@gmail.com}
  \phone{+33 6 88 59 08 87}
  % \mailaddress{IPREM Technopôle Hélioparc, 2 ave du Président P. Angot, FR-64053 Pau cedex 9}
  % \location{Pau, FRANCE}
  \homepage{gsalvatovallverdu.gitlab.io}
  % \twitter{@twitterhandle}
  \github{github.com/gvallverdu}
  %% You MUST add the academicons option to \documentclass, then compile with LuaLaTeX or XeLaTeX, if you want to use \orcid or other academicons commands.
  %\orcid{orcid.org/0000-0003-1116-8776}
  % \linkedin{linkedin.com/in/germain-salvato-vallverdu-398b31b2}
}

%% Make the header extend all the way to the right, if you want.
\begin{fullwidth}
\makecvheader
\vspace{-2mm}
\parbox{.7\paperwidth}{%
Inquisitive, exigent physical chemistry specialist with 10+ years experience in numerical
simulations and scientific programming. Seeking to leverage skills in communication,
management and computer science as HPC engineer.
% Critical Thinking, Active Learning,
% Je suis à la recherche d'un nouveau projet dans lequel je pourrai apporter mes compétences en simulation, programmation et outils numériques avec un regard original lié à ma formation de physico-chimiste. Curieux de nature, je sais m'adapter et me former pour relever de nouveaux challenges.
}

\end{fullwidth}
\vspace{-1mm}

%% Depending on your tastes, you may want to make fonts of itemize environments slightly smaller
\AtBeginEnvironment{itemize}{\small}

%% Provide the file name containing the sidebar contents as an optional parameter to \cvsection.
%% You can always just use \marginpar{...} if you do
%% not need to align the top of the contents to any
%% \cvsection title in the "main" bar.
% \marginpar{% \cvsection{My Life Philosophy}
%
% \begin{quote}
% ``Something smart or heartfelt, preferably in one sentence.''
% \end{quote}
%
% \cvsection{Most Proud of}
%
% \cvachievement{\faTrophy}{Fantastic Achievement}{and some details about it}
%
% \divider
%
% \cvachievement{\faHeartbeat}{Another achievement}{more details about it of course}
%
% \divider
%
% \cvachievement{\faHeartbeat}{Another achievement}{more details about it of course}

\cvsection{Skills}

\cvskill{Computer science}{5}

\divider

\cvskill{Mathematics / Modeling}{4}

\divider

\cvskill{Data Science}{4}

\divider

\cvskill{HPC}{4}

\divider

\cvsubsection{\color{accent}Computer science details}

\cvskill{Python}{5}
\smallskip
%\cvskill{Plotly/Dash}{3}
%\smallskip
\cvskill{Java}{2}
\smallskip
%\cvskill{Web Applications}{2}
%\smallskip
\cvskill{Linux/Unix/Bash}{4}
%\cvskill{MPI}{3}
\smallskip
\cvskill{C / Fortran}{4}
\smallskip
\cvskill{git}{3}
\smallskip
\cvskill{Machine Learning}{2}
%\smallskip
%\cvskill{git}{3}

\medskip

\cvtag{OOP}
\cvtag{Jupyter}
\cvtag{Plotly}\cvtag{Dash}
\cvtag{C++}
\cvtag{\LaTeX}
%\cvtag{git}
%\cvtag{Linux}\cvtag{bash}
\cvtag{MPI}
\cvtag{HTML/CSS}
\cvtag{Hugo/Jekyll}
% \cvtag{Jekyll}
\cvtag{Django}
\cvtag{Sphinx-doc}
%\cvtag{Java}
% \cvtag{Visual Studio Code}

\medskip

%\cvsubsection{\color{accent}Simulation codes}
%
%\cvtag{Lammps}
%\cvtag{VASP}
%\cvtag{Gromacs}
%\cvtag{Amber}
%\cvtag{Gaussian}
%\cvtag{Orca}
%\cvtag{VMD}

%\cvsubsection{Basic Knowledge}
% \cvtag{Hard-working}
% \cvtag{Eye for detail}
% \cvtag{Motivator \& Leader}

% \divider\smallskip
%
% \cvtag{C++}
% \cvtag{Embedded Systems}
% \cvtag{Statistical Analysis}

\cvsection{Languages}

\cvskill{French (native)}{5}

\divider

\cvskill{English}{4}

%\divider

%\cvskill{Spanish}{2}

%% Yeah I didn't spend too much time making all the
%% spacing consistent... sorry. Use \smallskip, \medskip,
%% \bigskip, \vpsace etc to make ajustments.
\medskip

\cvsection{Education}

%\cvevent{\small Ph.D.\ in physical-chemistry}{}{2006 -- 2009}{Université Paris-Sud 11}

\cvevent{\small Ph.D.\ in physical-chemistry}{}{2006 -- 2009}{Université Paris-Sud 11}
%Theoretical study of photophysics properties of fluorescent proteins

\divider

\cvevent{\small M.Sc.\ in Physical-Chemistry}{}{2004 -- 2006}{Université Paris-Sud 11}

\divider

% \cvevent{\small Magistère de Physico-Chimie Moléculaire}{}{2003 -- 2006}{Université Paris-Sud 11 ENS Cachan}
\cvevent{\small Magistère de Physico-Chimie Moléculaire}{}{2003 -- 2006}{Université Paris-Sud 11}
\vspace{-6pt}\hspace{3.5cm}{\color{body}\small ENS Cachan}
% \cvevent{\small Magistère de Physico-Chimie Moléculaire}{}{2003 -- 2006}{\shortstack{Université Paris-Sud 11\\ ENS Cachan}}

\divider

% \cvevent{\small B.Sc. \& CPGE\ in Physical-Chemistry}{Université Paris-Sud 11 / Lycée F. Arago}{2001 -- 2003}{}
%\vspace{3pt}
\cvevent{\small CPGE\ in Physics \& Chemistry}{}{2001 -- 2003}{Lycée Arago Perpignan}
}
% \cvsection{Experience}
\cvsection[sidebar]{Experience}

\cvevent{Associate Professor in numerical simulations}{University of Pau \& Pays Adour}{2010 -- Ongoing}{Pau, France}
\begin{itemize}
\item Develop original multi-scale computational strategies adapted to chemical systems in
multi-disciplinary teams
\item Develop and distribute data analyses libraries with Python
\item Supervise research projects with Ph.D. and Master students
\item Train UPPA users to HPC and programming languages using active learning
\end{itemize}
{\small 18 peer-reviewed articles, h-\textit{index} 8, 200 citations \hfill \hspace{-1ex}orcid.org/0000-0003-1116-8776}

\divider

\cvevent{Research Engineer}{CEA DAM}{2009 -- 2010}{Bruyères le châtel, France}
\begin{itemize}
\item Developed parallel, MPI-based, C rouines to extend HPC code
\item Accomplished the parametrization of coarse grain models for energetic materials
\end{itemize}

\divider

\cvevent{Ph.D. in physical-chemistry}{Université Paris Sud 11}{2006 -- 2009}{Orsay, France}
Theoretical study of photophysics properties of fluorescent proteins
\smallskip
\begin{itemize}
\item Developed novel numerical simulation strategies adapted to biological systems 
\item Elaborated Fortran routines to produce and analyze hugh throughput simulations
% \item Parametrization of new simulation models
\end{itemize}


\cvsection{Scientific Projects}

\cvevent{Energy storage materials -- Li-ion batteries}{}{}{}%{University of Pau \& Pays Adour / RS2E}{}{}
\vspace{-1mm}
\begin{itemize}
\item Surface reactivity of cathode materials
\item Coupled studies between quantum chemistry and experimental surface characterization
\item Interfaces: structure, electronic properties and thermochemistry
\end{itemize}

\divider

\cvevent{Petroleum chemistry and complex matrices}{}{}{}%{University of Pau \& Pays Adour / C2MC}{}{}
\vspace{-1mm}
\begin{itemize}
\item Physico-chemical properties of asphaltenes: solubility, aggregation
\item Molecular modelling and molecular dynamics simulations
\item Molecular characterization of complex matrices (heavy crude oil)
\end{itemize}

\divider

\cvevent{Python Scientific Libraries}{}{}{}%{University of Pau \& Pays Adour}{}{}
\vspace{-1mm}
\begin{itemize}
\item \texttt{Pymatgen}: Python Materials Genomics {(\footnotesize\url{http://pymatgen.org/team.html})}
\item LA-ICP MS: Chemical mapping analysis
{(\footnotesize\url{https://pychemapps.univ-pau.fr/icpms})}
\item Mammoth: a molecular force field optimizer \hbox{(\footnotesize\url{https://mammoth_uppa.gitlab.io/})}
\item Mosaïca: A generator and analyzer of nano-materials based on their intrinsec geometry 
\hbox{(\footnotesize\url{https://pychemapps.univ-pau.fr/mosaica/})}
\end{itemize}

% \divider

% \cvevent{PhD Thesis}{Université Paris-Sud 11}{3 years}{Orsay, France}
% Theoretical study of photophysics properties of fluorescent proteins




%% If the NEXT page doesn't start with a \cvsection but you'd
%% still like to add a sidebar, then use this command on THIS
%% page to add it. The optional argument lets you pull up the
%% sidebar a bit so that it looks aligned with the top of the
%% main column.
% \addnextpagesidebar[-1ex]{page3sidebar}


\end{document} 
